\documentclass[a0paper,fleqn]{betterposter}

%%%% Configuration (omitting most of the unused settings for brevity)



\renewcommand{\maincolumnbackgroundcolor}{methods}
\usepackage{caption}
\captionsetup{
	font=LARGE,       % set the font size to 'large' for both label and text
	labelfont=LARGE   % set the font size to 'large' for the label
}
\usepackage[square,numbers]{natbib}
\renewcommand{\bibfont}{\LARGE}
\begin{document}    
	\betterposter{
		%%%%%%%% SINGLE LEFT COLUMN (MERGED)
		
		\title{Application of the LISFLOOD-FP \\Hydrodynamic Model for Impact-Based Forecasting over the Eastern Africa \\Region}
		 \vspace{1em}%
		\section{
			\begin{minipage}[c]{0.07\textwidth}
				\includegraphics[width=\linewidth]{img/logo2.png}
			\end{minipage}%
		    \hspace{0.5em}%
			\begin{minipage}[c]{0.07\textwidth}
				\includegraphics[width=\linewidth]{img/logo1.png}
			\end{minipage}%
		    \hspace{1em}%
			\begin{minipage}[c]{0.9\textwidth}
				Jully Ouma$^{1,2}$, \underline{Nishadh Kalladath$^{1}$}, Khalid Hassaballah$^{1}$, Viola Otieno$^{1}$, Jason Kinyua$^{1}$, 
				\vspace{0.4em}%
				\\ Igbal Salah$^{1}$, Mohammed Hassan$^{1}$, Ahmed Amdihun$^{1}$ \& Guleid Artan$^{1}$
				\vspace{0.8em}%
				\\${^1}$ IGAD Climate Prediction and Applications Centre- ICPAC, Nairobi, Kenya
				\\${^2}$ United Nation Office for Disaster Risk Reduction, Africa Office, Kenya
			\end{minipage}
		}
	    \vspace{1em}%
		\hrule
		
		\section{Introduction}
		\begin{itemize}
			\item Impact-based forecasting (IBF) supports risk-oriented decisions in disaster risk management by emphasizing anticipatory actions that minimize damage and loss.
			\item Inundation forecasting is essential for IBF hazard modeling and use with impact functions, linking flood water depth to displacement risks and damage to infrastructure.
			\item Currently, IBF uses hydrological forecasts combined with historical maps for inundation data from stream flow forecasts, but it struggles to capture urban and riverine rainfall-induced floods. Although hydrodynamic modeling is precise, it is too computationally intensive for routine IBF.
			\item A parsimonious modeling framework that balances the complexity and simplicity of hydrodynamic processes might be a promising approach for generating inundation data.
			\item Current study evaluated the hydrodynamic model LISFLOOD-FP \cite{sharifian2023lisflood} for its potential in impact-based forecasting. We also assessed the operational IBF suitability of the parsimonious, simplified model RIM2D \cite{apel2022brief}.
			
		\end{itemize}
		
		\section{Methods}
		\\ The Figure shows the method used to compare LISFLOOD-FP and RIM2D for operational use in cloud computing.
		\begin{center}
		%\begin{figure}[h!]	
			\includegraphics[width=\textwidth]{img/rim2d-v2.png}
			\captionof{figure}{\LARGE{The GPU versions of the models are used in a similar GPU cloud computing setup.}}
		%\end{figure}
		\end{center}
		
	    \section{Results}
		\begin{itemize}
			\item Test case simulations indicate that RIM2D is more efficient for operational IBF due to its parsimonious hydrological modeling approach, making it suitable for IBF risk measures.
			\item The validation of RIM2D's hydrodynamic modeling in East Africa is ongoing; the supporting materials detail the Python programs and cloud setup for the model.
			%\item Study is ongoing to extend the application
		\end{itemize}
		%% Institution logo
		%\includegraphics[width=\textwidth]{img/logo}\\
		\section{References}
		%\begin{enumerate}
		%\item 
		%\item OpenAI. "GPT-4 Technical Report." arXiv e-prints, 2023, arXiv:2303.08774.
		%\item Ankan, Ankur, and Abinash Panda. 2015. [4]. 
		%\item Ankan, Ankur, and Abinash Panda. 2015. [4] \LARGE{"pgmpy: Probabilistic graphical models using python." Proceedings of the 14th python in science conference (scipy 2015). Vol. 10. Citeseer, 2015.}
		%\item \url{github.com/nishadhka/bn-ibf/code/01-simple-bn-flood.ipynb }
		\renewcommand{\bibsection}{}                                                                                                               
		\bibliographystyle{unsrtnat} % or another suitable style like abbrvnat, unsrtnat, etc.
		\bibliography{references} % if your file is named references.bib
		
		
	    %\end{enumerate}
		
		%\author{Mike Morrison}
		%\author{Rafael Bailo}
		%\institution{Optional Institution Under Name}
	}	
	{
		%%%%%%%% MAIN COLUMN
		\maincolumn{
			%%%% Main space
						
		\textbf{RIM2D}, a streamlined, \textbf{parsimonious} version of \textbf{LISFLOOD-FP} model, might be the ideal choice for \textbf{impact-based forecasting}. After all, why putting on a \textbf{space suit} to water the \textbf{garden}?
			
		}{
			%%%% Bottom space
			\qrcode{img/qrcode-rim2d}{img/smartphoneWhite}{
				\textbf{Scan the QR Code} for \\ supporting materials
				\\ \textbf{@ GitHub Repository:}
				\\icpac-igad/rim2d-ibf
				\\\textbf{For comments and questions}
				\\icpac-igad/rim2d-ibf/issues
			}
		}
		
	}
	
\end{document}
